\documentclass[a4paper]{deedy-resume} 
% \documentclass[letterpaper]{deedy-resume} 
\usepackage{fixltx2e}
\usepackage{hyperref}
\usepackage{color}
\usepackage[document]{ragged2e}

\begin{document}
\justify

%----------------------------------------------------------------------------------------
%	TITLE SECTION
%----------------------------------------------------------------------------------------

\begin{center}
%	NAME
\Huge \textbf{JIAJUN ZHANG}\\
\end{center}
\smallskip
%
%----------------------------------------------------------------------------------------
%	LEFT CONTACT
%----------------------------------------------------------------------------------------
%
\begin{minipage}[t]{0.49\textwidth} 
\textbf{Address:} \color{gray} 526 W 123 ST, New York, 10027 \\
\textbf{Email:} \href{mailto:z.jiajun@colum�bia.edu} {\url{ z.jiajun@columbia.edu}} \\
\textbf{Phone:} 917-640-3324
\end{minipage} 
\hfill
%
%----------------------------------------------------------------------------------------
%	RIGHT CONTACT
%----------------------------------------------------------------------------------------
%
\begin{minipage}[t]{0.50\textwidth} 
\textbf{Webpage: }\color{gray} \url{http://petercanmakeit.com}\\
\textbf{LinkedIn: }\url{http://linkedin.com/in/zhang-jiajun}\\
\textbf{GitHub: } \url{https://github.com/petercanmakit}
\end{minipage}
\smallskip


%----------------------------------------------------------------------------------------
%	LEFT COLUMN
%----------------------------------------------------------------------------------------

\begin{minipage}[t]{0.29\textwidth} 

\section{Objective} 
\descript{Software Development Engineering, Full-time}
\sectionspace
\sectionspace

\section{Education} 

\subsection{Columbia University}

\descript{M.S. Computer Engineering}
\location{Expected Dec 2017 | GPA: 3.54}

\sectionspace 

\subsection{Zhejiang University}

\descript{B.E. Information and Communication Engineering}
\location{June 2016 | GPA: 3.72}

\sectionspace 

\sectionspace 

\section{Coursework}

\subsection{Graduate}

Operating Systems\\
Analysis of Algorithms\\
Computer Networks\\
Database Systems Implementation\\
Big Data Analytics

\sectionspace 

\subsection{Undergraduate}

Functional Programming\\
Data Structures\\
Computer Architecture\\
Theory of Probability\\

\sectionspace 

\sectionspace

\section{Skills}

\subsection{Languages}
Java \textbullet{} Python \textbullet{} C \textbullet{} SQL
\sectionspace

\subsection{Back end}
Flask \textbullet{} PostgreSQL \textbullet{} SQLAlchemy
\sectionspace 

\subsection{Front end}
HTML \textbullet{} CSS  \textbullet{}  JavaScript \textbullet{} jQuery
\sectionspace 

\subsection{Other}
Linux \textbullet{} Git \textbullet{} Google Cloud Platform
\sectionspace 
\sectionspace

\section{Awards}
\vspace{0.2cm}
\setlength{\leftmargini}{1em}
\begin{tightitemize}
\item First Prize in the National Undergraduate Electronic Design Contest, Zhejiang Prov. | 2015
\item ISEE Texas Instruments College Student Grant, Zhejiang University | 2014-2015
\end{tightitemize}


\sectionspace


\sectionspace 

\end{minipage} 
\hfill
%
%----------------------------------------------------------------------------------------
%	RIGHT COLUMN
%----------------------------------------------------------------------------------------
%
\begin{minipage}[t]{0.66\textwidth} 

\section{Work Experience}

\runsubsection{Full Stack Developer} |\descript{\small Interactive Pedestrian Injury Mapper Web App, \color{gray}\url{https://petercanmakit.github.io/IPIM/} }

\location{May 2017 - Aug 2017, Columbia University Medical Center, New York, NY}
\vspace{\topsep} 
\begin{tightitemize}
\item Used \textbf{Google Maps} to develop an interface for victims to visualize the route on which they were hit by a vehicle
\item Built a questionnaire view to collect victims' information, and embedded methods for monitoring the user behavior
\item Worked with \textbf{PostGIS} extension on \textbf{PostgreSQL} for location storing
\item Created an admin interface to retrieve data and provide the statistics about the datasets using \textbf{Chart.js}, and to cluster the accident spots on the map
\item Built a wrapper (\href{https://github.com/petercanmakit/gapy}{\color{gray}\url{Gapy}}) for \textbf{Google Analytics} to retrieve page views and event tracker information so that it makes constructions on \textbf{Flask} server app easier
\end{tightitemize}


\sectionspace

\section{Project Experience}

\descript{UdpChat [Java, Socket Programming]}

\location{Feb 2017 - Mar 2017, Columbia University}
\begin{tightitemize}
\item Developed a P2P chat program with functionalities of online / offline chatting
\item Built the server as it broadcasts the contact information of all users and manages messages for offline users
\item Applied acknowledgment messages to provide reliable communication
\end{tightitemize}
\sectionspace 

\descript{HTTP Server [Linux, C , Socket Programming]}

\location{Jan 2017 - Feb 2017, Columbia University}
\begin{tightitemize}
\item Built a web server which handles HTTP requests, using socket programming
\item Starting from single process, developed to multiple processes and threads in order to increase throughput
\end{tightitemize}
\sectionspace 

\descript{Other Projects}
\begin{tightitemize}
\item Built an image processing webpage (\href{https://petercanmakit.github.io/imgProc/}{\color{gray}\url{imgProc}}) [JavaScript]
\item Created a music sharing \textbf{Web App} [Python, SQL, Google Cloud, Flask] 
\item Built a linear \textbf{File System} on loop devices [Linux Kernel, C]
\item Implemented a Random-Robin \textbf{Task Scheduler} [Linux Kernel, C]
\item Wrote a program to simulate Go-Back-N Transfer Protocol and Distance Vector Routing Algorithm [Java, Socket Programming]
\item Created a research tool for motor collision analysis [Hadoop, Pyspark]
\end{tightitemize}
\sectionspace 


\section{Research Experience}

\runsubsection{Teaching Assistant} |\descript{\small CSEE 4119 Computer Networks}

\location{Sep 2017 - present, Columbia University, New York, NY}
\begin{tightitemize}
\item Provide weekly individual instruction and guidance to students]
\item Cooperate with the TA team to help the professor assess exams, written assignments  and programming projects
\end{tightitemize}

\sectionspace 

\runsubsection{Research Assistant} |\descript{\small Enhancement of the Palmprint Directional Field}

\location{Nov 2015 - May 2016, Zhejiang University, Hangzhou, China}
\begin{tightitemize}
\item Utilized \textbf{OpenCV} to extract the directional field and preprocess it
\item Implemented a Random Forest algorithm with \textbf{scikit-learn} to enhance the palmprint directional field
\item Wrote a \textbf{Python} visualization tool to analyze the enhanced directional field
\end{tightitemize}

\sectionspace

\sectionspace


\end{minipage}

\end{document}