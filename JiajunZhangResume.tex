\documentclass[a4paper]{deedy-resume} 
\usepackage{fixltx2e}
\usepackage{hyperref}
\usepackage{color}
\usepackage[document]{ragged2e}

\begin{document}

%----------------------------------------------------------------------------------------
%	TITLE SECTION
%----------------------------------------------------------------------------------------

\begin{center}
\Huge \textbf{JIAJUN ZHANG}\\*
\begin{normalsize}
\color{darkgray} 
\textbf{Address:} \color{gray} 526 W 123 ST, New York, 10027\\
\textbf{Webpage:} \url{http://petercanmakeit.com} | \textbf{Email:} \href{mailto:z.jiajun@colum�bia.edu} {\url{ z.jiajun@columbia.edu}} | \textbf{Phone:} (917)-640-3324 \\

\textbf{LinkedIn:} \url{http://linkedin.com/in/zhang-jiajun} | \textbf{GitHub:} \url{https://github.com/petercanmakit}

\end{normalsize}
\end{center}

%----------------------------------------------------------------------------------------
%	LEFT COLUMN
%----------------------------------------------------------------------------------------

\begin{minipage}[t]{0.29\textwidth} 

\section{Objective} 
\descript{Software Development Engineering, Full-time}
\sectionspace
\sectionspace

\section{Education} 

\subsection{Columbia University}

\descript{M.S. Computer Engineering}
\location{Expected Dec 2017 | GPA: 3.54}

\sectionspace 

\subsection{Zhejiang University}

\descript{B.E. Information and Communication Engineering}
\location{June 2016 | GPA: 3.72}

\sectionspace 

\sectionspace 

\section{Coursework}

\subsection{Graduate}

Operating Systems\\
Analysis of Algorithms\\
Computer Networks\\
Database Systems Implementation\\
Big Data Analytics

\sectionspace 

\subsection{Undergraduate}

Functional Programming\\
Data Structures\\
Computer Architecture\\
Theory of Probability\\

\sectionspace 

\sectionspace

\section{Skills}

\subsection{Languages}
Java \textbullet{} Python \textbullet{} C \textbullet{} SQL
\sectionspace

\subsection{Back end}
Flask \textbullet{} PostgreSQL \textbullet{} SQLAlchemy
\sectionspace 

\subsection{Front end}
JavaScript \textbullet{} HTML \textbullet{} CSS \textbullet{} JQuery
\sectionspace 

\subsection{Other}
Linux \textbullet{} Git \textbullet{} Google Cloud Platform
\sectionspace 
\sectionspace

\section{Awards}

\textbullet{} First Prize in the National Undergraduate Electronic Design Contest, Zhejiang Prov. | 2015
\\
\textbullet{} ISEE Texas Instruments College Student Grant, Zhejiang University | 2014-2015

\sectionspace


\sectionspace 

\end{minipage} 
\hfill
%
%----------------------------------------------------------------------------------------
%	RIGHT COLUMN
%----------------------------------------------------------------------------------------
%
\begin{minipage}[t]{0.66\textwidth} 

\section{Experience}

\runsubsection{Full Stack Developer} |\descript{\small Interactive Pedestrian Injury Mapper Web App, \url{https://petercanmakit.github.io/IPIM/} }

\location{May 2017 - Aug 2017, Columbia University Medical Center, New York, NY}
\vspace{\topsep} 
\begin{tightitemize}
\item Used \textbf{Google Maps} to develop an interface for victims to visualize the route on which they were hit by a vehicle.
\item Built a questionnaire view to collect victims’ information. Implemented monitoring of the user behavior when answering.
\item Worked with \textbf{PostGIS} extension on \textbf{PostgreSQL} for location storing.
\item Created an admin interface to retrieve data and display the statistics about the datasets with \textbf{Chart.js} and clusters the spots of accidents on maps.
\item Used \textbf{Flask} to construct the server app. Built a wrapper (\href{https://github.com/petercanmakit/gapy}{\url{Gapy}}) for \textbf{Google Analytics} to retrieve page views and event tracker information.
\end{tightitemize}

\sectionspace 

\runsubsection{Teaching Assistant} |\descript{\small CSEE 4119 Computer Networks}

\location{Sep 2017 - present, Columbia University, New York, NY}
\begin{tightitemize}
\item Provide weekly individual instruction and guidance to students. Answer questions asked by students after class.
\item Cooperate with the professor and the TA team to assess the written assignments, the programming projects, and the exams.
\end{tightitemize}

\sectionspace 

\runsubsection{Research Assistant} |\descript{\small Enhancement of the Palmprint Directional Field}

\location{Nov 2015 - May 2016, Zhejiang University, Hangzhou, China}
\begin{tightitemize}
\item Utilized \textbf{OpenCV} to extract the directional field and preprocess it.
\item Implemented a Random Forest algorithm with \textbf{scikit-learn} to enhance the palmprint directional field.
\item Wrote a \textbf{Python} visualization tool to analyze the enhanced directional field.
\end{tightitemize}

\sectionspace

\sectionspace

\section{Projects}

\descript{UdpChat [Java, Socket Programming]}

\location{Feb 2017 - Mar 2017, Columbia University}
\begin{tightitemize}
\item Developed a P2P chat program with functionalities of online / offline chatting.
\item Built the server as it broadcasts the contact information of all users and receives / sends messages for offline users.
\item Applied acknowledgment messages to provide reliable communication.
\end{tightitemize}
\sectionspace 

\descript{HTTP Server [Linux, C , Socket Programming]}

\location{Jan 2017 - Feb 2017, Columbia University}
\begin{tightitemize}
\item Built a web server which handles HTTP requests, using socket programming with C language on Linux.
\item Starting from single process, developed to multiple processes and threads in order to increase throughput.
\end{tightitemize}
\sectionspace 

\descript{Other Projects}
\begin{tightitemize}
\item Created a Music Sharing Web App [Python, SQL, Google Cloud, Flask] 
\item Implemented a Linear File System on loop devices [Linux Kernel, C]
\item Implemented a Random-Robin Task Scheduler [Linux Kernel, C]
\item Wrote a program to simulate Go-Back-N Transfer Protocol and Distance Vector Routing Algorithm [Java, Socket Programming]
\item Created a Research Tool for motor collision analysis [Hadoop, Pyspark]
\end{tightitemize}
\sectionspace 


\end{minipage}

\end{document}